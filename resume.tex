%!TEX TS-program = xelatex

% Copied from my friend Sheng's repository here:
% https://github.com/sheng0501/CV. It looks like he downloaded this
% latex template from
% https://www.latextemplates.com/template/plasmati-graduate-cv

\documentclass[a4paper,10pt]{article}

%A Few Useful Packages
\usepackage{marvosym}
\usepackage{xunicode,xltxtra,url,parskip} 	%other packages for formatting
\usepackage{fontspec} 					%for loading fonts
\RequirePackage{color,graphicx}
\usepackage[usenames,dvipsnames]{xcolor}
%% \usepackage[big]{layaureo} 				%better formatting of the A4 page
% an alternative to Layaureo can be ** \usepackage{fullpage} **
\usepackage{fullpage}
\usepackage{supertabular} 				%for Grades
\usepackage{titlesec}					%custom \section

%Setup hyperref package, and colours for links
\usepackage{hyperref}
\usepackage{tikz}

\definecolor{linkcolour}{rgb}{0,0.2,0.6}
\hypersetup{colorlinks,breaklinks,urlcolor=linkcolour, linkcolor=linkcolour}

%FONTS
\defaultfontfeatures{Mapping=tex-text}
%\setmainfont[SmallCapsFont = Fontin SmallCaps]{Fontin}
%%% modified for Karol Kozioł for ShareLaTeX use
\setmainfont[
SmallCapsFont = Fontin-SmallCaps.otf,
BoldFont = Fontin-Bold.otf,
ItalicFont = Fontin-Italic.otf
]
{Fontin.otf}
%%%

%CV Sections inspired by: 
%http://stefano.italians.nl/archives/26
\titleformat{\section}{\Large\scshape\raggedright}{}{0em}{}[\titlerule]
\titlespacing{\section}{0pt}{1pt}{1pt}
%Tweak a bit the top margin
% \addtolength{\voffset}{-2.0cm}
\addtolength{\oddsidemargin}{-.5in}
\addtolength{\evensidemargin}{-.5in}
\addtolength{\textwidth}{1.0in}

\addtolength{\topmargin}{-.575in}
\addtolength{\textheight}{1.75in}

%Italian hyphenation for the word: ''corporations''
\hyphenation{im-pre-se}

%-------------WATERMARK TEST [**not part of a CV**]---------------
\usepackage[absolute]{textpos}

\setlength{\TPHorizModule}{30mm}
\setlength{\TPVertModule}{\TPHorizModule}
\textblockorigin{2mm}{0.65\paperheight}
\setlength{\parindent}{0pt}

%--------------------BEGIN DOCUMENT----------------------
\begin{document}

%WATERMARK TEST [**not part of a CV**]---------------
%\font\wm=''Baskerville:color=787878'' at 8pt
%\font\wmweb=''Baskerville:color=FF1493'' at 8pt
%{\wm 
%	\begin{textblock}{1}(0,0)
%		\rotatebox{-90}{\parbox{500mm}{
%			Typeset by Alessandro Plasmati with \XeTeX\  \today\ for 
%			{\wmweb \href{http://www.aleplasmati.comuv.com}{aleplasmati.comuv.com}}
%		}
%	}
%	\end{textblock}
%}

\pagestyle{empty} % non-numbered pages

\font\fb=''[cmr10]'' %for use with \LaTeX command

%--------------------TITLE-------------
\par{\centering
		{\Huge Lucas \textsc{Groenendaal}
  }\par}

\par{\centering
  {\normalsize
  914-588-8612 | groenendaal92@gmail.com | https://github.com/lag13
}\par}
\bigskip

%--------------------SECTIONS-----------------------------------
\section{Objective}
Help people, learn new things, and enjoy my time doing it.

\section{Skills}
\begin{itemize}
  % \small
  \item Languages: Golang, Clojure, PHP
  \item Tools/Infrastructure/Software: Docker, Terraform, Ansible, Jenkins, Kubernetes, Prometheus, Vault, Elasticsearch, MySQL, Redis, Grafana, and various AWS services
\end{itemize}

\section{Work Experience}
\begin{tabular}{r|p{13cm}}
 \emph{Current} & Site Reliability Engineering Manager at \textsc{Guaranteed Rate} \\
 \textsc{Oct 2020} &
 \footnotesize{Continued my previous tasks from when I was on the DevOps team plus mentored a developer (who was fresh out of college) to the point where they became an actively contributing member of the DevOps team.} 
 \\
 \multicolumn{2}{c}{} \\ 
 \textsc{Oct 2020} & DevOps Engineer at \textsc{Guaranteed Rate} \\
 \textsc{Feb 2019} &
 \footnotesize{Maintained, upgraded, and standardized infrastructure used by the development organization while fielding support requests from our developers.} 
 \begin{itemize}
 \item \footnotesize{Deployed an upgraded version of Elasticsearch (v5 -> v7). Benefits: improved stability, consolidated multiple elasticsearch deployments into just this one, added alerting, wrote documentation on how to upgrade (which a teammate successfully followed!)}
 \item \footnotesize{Fixed an intermittent DNS issue on our old 1.8 k8s clusters which was causing stability issues}
 \item \footnotesize{Conducted technical and behavioral interviews for potential candidates}
 \item \footnotesize{Sampling of miscellaneous requests from the IT organization:}
   \begin{itemize}
   \item \footnotesize{Developer asked why their lambda was failing to talk to the outside world. Turns out they had selected an AWS security group with no egress rules}
   \item \footnotesize{Developer requested to pair on writing terraform for their infrastructure}
   \item \footnotesize{Developer asked why their DNS record was not resolving and it was because they made a CNAME record pointing to an IP address when it should be an A type record}
   \item \footnotesize{Developer requested help with refining AWS IAM permissions}
   \item \footnotesize{Security asked if an EC2 instance could be terminated. I looked at VPC flow logs to make this determination}
   \end{itemize}
 \end{itemize}
 \\
 \multicolumn{2}{c}{} \\
 \textsc{Feb 2019} & Software Engineer II at \textsc{CareerBuilder} \\
 \textsc{Jan 2017} &
 \footnotesize{Continued developing new features and maintaining the infrastructure around our applicant tracking system (ATS)}
 \begin{itemize}
   \item \footnotesize{Traveled to India to train the new development team so they could work on our ATS.}
   \item \footnotesize{Created microservices to expand the ability of the ATS (Cangrade assessments)}
   \item \footnotesize{Cleaned up technical debt (upgraded golang versions for our APIs, upgraded our software to use CircleCI 2.0, refactored older projects to match the structure of newer ones, etc...)}
 \end{itemize}
 \\
 \multicolumn{2}{c}{} \\
 \textsc{Jan 2017} & Software Engineer I at \textsc{CareerBuilder} \\
 \textsc{Oct 2015} &
 \begin{itemize}
   \item \footnotesize{Developed software which managed our applicant tracking system's (ATS) life cycle (creation, deployment, upgrades, deletions)}
   \item \footnotesize{Created a feature flag API, which was consumed by the ATS, to enable faster development of features while reducing the number of bugs that end users see}
 \end{itemize}
 \\
 \multicolumn{2}{c}{} \\
 \textsc{Oct 2015} & Integration Software Developer at \textsc{CareerBuilder} \\
 \textsc{Aug 2014} &
 \footnotesize{Edited XML files to configure certain behaviors of our applicant tracking system (ATS). Ended up learning and writing vimscript (thanks https://learnvimscriptthehardway.stevelosh.com/!!) to automate some of the more repetitive configurations}
 \\
 \multicolumn{2}{c}{} \\
\end{tabular}

\section{Education}
\begin{tabular}{rl}	
  \textsc{2014} & B.A. in Mathematics and Computer Science at SUNY Geneseo \\
\end{tabular}

\section{About me}
I enjoy doing most things as long as I have good people to do them with! Some interesting things I've done:
\begin{itemize}
  \item Ran the 2019 Chicago marathon
  \item Jumped out of a perfectly good airplane (i.e. skydived)
  \item Raced on a sailboat during the summer of 2020
\end{itemize}

On weekends you might find me cooking while listening to movie soundtracks or playing guitar.

\end{document}
